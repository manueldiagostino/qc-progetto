\section{Background}\label{sec:background}

\subsection{Soluzioni classiche}
Le tecniche classiche per la rilevazione dei contorni prevedono l'utilizzo di 
kernel specifici, che permettono di calcolare nuovi valori di intensità per i pixel 
dell'immagine. Tra i metodi più famosi vi è sicuramente l'operatore di Sobel, 
che si può descrivere tramite l'applicazione di 2 kernel all'immagine originale:
\[
\mathbf{G_x} =
\begin{bmatrix}
+1 & 0 & -1 \\
+2 & 0 & -2 \\
+1 & 0 & -1
\end{bmatrix}
, \, 
\mathbf{G_y} =
\begin{bmatrix}
+1 & +2 & +1 \\
0 & 0 & 0 \\
-1 & -2 & -1
\end{bmatrix}
\]

Un'altra opzione, forse tra le più utilizzate al giorno d'oggi, è l'operatore di Canny.
Questo metodo ha un funzionamento del tutto analogo al precedente, ma aggiunge 
meccanismi per la riduzione del rumore nell'immagine \cite{digital_image_processing}.
La complessità di queste tecniche è lineare rispetto al numero di pixel totali 
dell'immagine, dato che è necessaria una visita completa. 

Per un'immagine $M \times L = N$, si utilizzano $n$ bit per enumerare i pixel dell'immagine
in formato binario, dove $N = 2^n$, ottenendo una complessità rispetto ai bit esponenziale $\bigo{2^n}$.
In questo progetto verrà mostrato come, dopo una prima fase di preparazione, è possibile risolvere
il problema in tempo costante $\bigo{1}$.

\subsection{Sistemi quantistici}

Analogamente a quanto accade nei computer classici, i computer quantistici utilizzano i \textbf{quantum bit}, chiamati \emph{qubit}. I qubit rappresentano la più piccola unità di informazione e sono implementati attraverso sistemi quantistici bidimensionali. Le quantità fisiche comunemente usate per questo scopo includono lo spin di una particella o gli stati eccitati degli atomi.

Assemblando più qubit, è possibile costruire sistemi quantistici la cui dinamica è descritta da spazi vettoriali complessi. Un sistema composto da un singolo qubit è completamente descritto da
\begin{equation}
	\ket{\psi} = \begin{bmatrix}
		\alpha\\ \beta
	\end{bmatrix} = \alpha \ket{0} + \beta \ket{1},\quad \alpha,\beta \in
	\mathbb{C}
	\label{eq:1-qubit-state}
\end{equation}

Mentre un bit classico può assumere soltanto uno tra due possibili valori
(generalmente 0 e 1), un bit quantistico è denotato da una combinazione lineare
dei suoi stati base, pesata dai coefficienti complessi $\alpha$ e $\beta$. Tali
coefficienti sono detti \emph{ampiezze di probabilità} e rispettano la seguente:
\begin{equation}
	|\alpha|^2 + |\beta|^2 = 1
	\label{eq:qubit-prob}
\end{equation}

Per descrivere lo stato di un sistema quantistico composto da più qubit, è
necessario effettuare un'operazione chiamata \emph{prodotto tensoriale} tra i
singoli stati coinvolti. Ad esempio, considerati i vettori di stati
\[
\ket{\psi_1} = \begin{bmatrix} a_1 \\ a_2 \end{bmatrix}, \quad \ket{\psi_2} = \begin{bmatrix} b_1 \\ b_2 \end{bmatrix},
\]
il loro prodotto tensore è definito come:
\begin{equation}
\ket{\psi_1} \otimes \ket{\psi_2}
= \begin{bmatrix}
a_1 \begin{bmatrix} b_1 \\ b_2 \end{bmatrix} \\
a_2 \begin{bmatrix} b_1 \\ b_2 \end{bmatrix}
\end{bmatrix} =
\begin{bmatrix}
a_1 b_1 \\ a_1 b_2 \\ a_2 b_1 \\ a_2 b_2
\end{bmatrix}.
\label{eq:tensor-prod}
\end{equation}
Equivalentemente, può essere scritto come $\ket{\psi_1}\ket{\psi_2} \text{ o } \ket{\psi_1
\psi_2}$.

\todo{Se serve, parlare dell'entanglement}

\subsection{Circuiti quantistici}

Analogamente a quanto accade nei circuiti digitali classici, i circuiti
quantistici eseguono calcoli manipolando le informazioni immagazzinate nei
qubit. Questo viene realizzato attraverso dispositivi chiamati \textbf{quantum
gate} (\emph{porte quantistiche}), che sono l'equivalente quantistico delle porte logiche classiche
ma operano secondo i principi della meccanica quantistica. L'applicazione di una
matrice \emph{complessa unitaria} ad uno stato quantistico modella matematicamente l'azione di un
gate su di esso. Formalmente, data $U \in \mathcal{M}_{n \times n}(\mathbb{C})$
l'unitaria associata ad una porta logica e dato lo stato $\ket{\psi} \in
\mathbb{C}^n$, lo stato risultante dall'applicazione è definito come:
\begin{equation}
	\ket{\psi'} = U\ket{\psi}
	\label{eq:gate-res-state}
\end{equation}

\subsection{Quantum Image Processing}

La \textbf{Quantum Image Processing} (\emph{processamento quantistico
dell'immagine}) si concentra sullo sviluppo di algoritmi in grado di codificare immagini all’interno di circuiti quantistici e di processarle utilizzando operazioni quantistiche. 

\subsubsection*{Rappresentazione delle immagini} Tra le varie tecniche proposte negli
ultimi anni, la \textbf{Quantum Probability Image Encoding (QPIE)} \cite{qpie} utilizza le ampiezze di probabilità di uno stato quantistico per memorizzare i valori dei pixel di un'immagine classica. Dati $n$ qubit, essa consente di rappresentare un'immagine in toni di grigio di $2^n$ pixel tramite una superposizione di stati. In generale, il numero di qubit necessari è calcolato tramite:
\begin{equation}
	n = \lceil{\log_2{N}}\rceil
	\label{eq:qpie-n-qubit}
\end{equation}

Come mostrato in Fig.
\ref{fig:bw-4x4-img},
ogni pixel può essere numerato utilizzando stringhe binarie ($00,01,10,11$);
l'intera immagine è quindi
rappresentabile come una matrice $2\times2$ delle intensità di colore. In questa
notazione, il singolo termine $I_{i}$ corrisponde all'intensità del pixel in posizione
$(x,y)$ (rispetto all'angolo in alto a destra), tale per cui $i = {xy}_{10}$.
\begin{figure}[h!]
    \centering
    \includegraphics[width=0.5\columnwidth]{classical_repr.png}
		\caption{Rappresentazione di un'immagine B\&W 4x4 pixel.}
    \label{fig:bw-4x4-img}
\end{figure}

Per rappresentare l'immagine come una superposizione di stati base, è necessario
che venga rispettata l'Eq. \ref{eq:qubit-prob}; bisogna infatti normalizzare le singole intensità come segue:
\begin{equation}
	c_i = \frac{I_i}{\sqrt{\sum_{k}^{}{I_{k}^2}}}
	\label{eq:intensities-norm}
\end{equation}
In Fig. \ref{fig:bw-4x4-img-qpie}
viene mostrato il risultato della normalizzazione.

\begin{figure}[h!]
    \centering
    \includegraphics[width=0.5\columnwidth]{QPIE_repr.png}
		\caption{Rappresentazione della Fig. \ref{fig:bw-4x4-img} tramite QPIE.}
    \label{fig:bw-4x4-img-qpie}
\end{figure}



\subsubsection*{QHED}

\subsection{Modellazione del rumore}
