\section{Background}\label{sec:background}

\subsection{Soluzioni classiche}
Le tecniche classiche per la rilevazione dei contorni prevedono l'utilizzo di 
kernel specifici, che permettono di calcolare nuovi valori di intensità per i pixel 
dell'immagine. Tra i metodi più famosi vi è sicuramente l'operatore di Sobel, 
che si può descrivere tramite l'applicazione di 2 kernel all'immagine originale:
\[
\mathbf{G_x} =
\begin{bmatrix}
+1 & 0 & -1 \\
+2 & 0 & -2 \\
+1 & 0 & -1
\end{bmatrix}
, \, 
\mathbf{G_y} =
\begin{bmatrix}
+1 & +2 & +1 \\
0 & 0 & 0 \\
-1 & -2 & -1
\end{bmatrix}
\]

Un'altra opzione, forse tra le più utilizzate al giorno d'oggi, è l'operatore di Canny.
Questo metodo ha un funzionamento del tutto analogo al precedente, ma aggiunge 
meccanismi per la riduzione del rumore nell'immagine \cite{digital_image_processing}.
La complessità di queste tecniche è lineare rispetto al numero di pixel totali 
dell'immagine, dato che è necessaria una visita completa. 

Per un'immagine $M \times L = N$, si utilizzano $n$ bit per enumerare i pixel dell'immagine
in formato binario, dove $N = 2^n$, ottenendo una complessità rispetto ai bit esponenziale $\bigo{2^n}$.
In questo progetto verrà mostrato come, dopo una prima fase di preparazione, è possibile risolvere
il problema in tempo costante $\bigo{1}$.