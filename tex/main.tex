\documentclass[journal]{IEEEtran}
\renewcommand{\IEEEkeywordsname}{Keywords}

\usepackage[italian]{babel}
\usepackage[utf8]{inputenc}
\usepackage[T1]{fontenc}

\setlength{\marginparwidth}{2cm}
\usepackage[italian, textsize=small, textwidth=2cm]{todonotes}

\usepackage{hyperref}
\usepackage{url}
\usepackage{tikz}
\usetikzlibrary{shapes.geometric, arrows}
\usepackage[dvipsnames]{xcolor}
\usepackage{listings}
\renewcommand{\lstlistingname}{Codice}

\usepackage{amsmath}
\newcommand{\bigo}[1]{\ensuremath{O\left({#1}\right)}}
\usepackage{booktabs}

\begin{document}

\title{Rilevamento quantistico dei bordi}

\author{\IEEEauthorblockN{Manuel Di Agostino}
\IEEEauthorblockA{\textit{Università degli studi di Parma} \\
Parma, Italia\\
manuel.diagostino@studenti.unipr.it}\\

\IEEEauthorblockN{Leonardo Ongari}
\IEEEauthorblockA{\textit{Università degli studi di Parma} \\
Cremona, Italia \\
leonardo.ongari@studenti.unipr.it}
}

\maketitle

\begin{abstract}
	Il rilevamento dei bordi è un processo fondamentale nell'estrazione delle caratteristiche di un'immagine ed è ampiamente utilizzato per analizzare la struttura degli oggetti rappresentati. Tuttavia, con l'aumento della risoluzione delle immagini, i metodi classici affrontano significative sfide computazionali a causa delle operazioni pixel-per-pixel necessarie. Il \emph{Quantum Image Processing} (QIP), offre il potenziale per accelerazioni esponenziali in determinati scenari, sfruttando algoritmi e rappresentazioni quantistici. Questo articolo esplora l'applicazione dell'algoritmo \emph{Quantum Hadamard Edge Detection} (QHED), implementato utilizzando la rappresentazione \emph{Quantum Probability Image Encoding} (QPIE). Utilizzando i principi quantistici e il framework Qiskit, si analizzano i vantaggi e le prospettive di questo nuovo approccio per il rilevamento dei bordi.
\end{abstract}

\begin{IEEEkeywords}
Rilevamento dei bordi, Quantum computing, Sobel.
\end{IEEEkeywords}

\section{Introduzione}

L'identificazione dei bordi è una tecnica fondamentale nell'elaborazione delle
immagini, utilizzata per individuare i contorni degli oggetti e le variazioni di
intensità in una scena. Questa metodologia rappresenta una componente cruciale
in numerosi ambiti, dalla computer vision alla robotica, fino all'analisi medica
delle immagini. Nonostante i progressi significativi nell'elaborazione classica
delle immagini, l'aumento della risoluzione e della complessità dei dati visivi
ha portato a sfide computazionali sempre maggiori, rendendo spesso i metodi
tradizionali onerosi in termini di tempo e risorse.

Nei primi anni '60, i filtri di Sobel \cite{SobelFeldman1968IsotropicGradient} e Prewitt furono introdotti come i primi metodi strutturati per il rilevamento dei bordi. Entrambi basati su operatori convolutivi, questi algoritmi utilizzano maschere discrete per approssimare il gradiente di intensità in un'immagine, rilevando così variazioni significative nei livelli di grigio. Sebbene semplici ed efficienti, essi risultano sensibili al rumore e con conseguente difficoltà nel gestire bordi sfumati. Negli anni '80, l'algoritmo di Canny~\cite{CannyPaper} rappresentò una svolta significativa grazie all'introduzione di un approccio più sofisticato al rilevamento dei bordi; ancora oggi, rimane uno tra i metodi più utilizzati. Con l'avanzare della tecnologia e l'aumento della potenza computazionale, il rilevamento dei bordi ha beneficiato dell'utilizzo di tecniche basate sull'intelligenza artificiale, come le \emph{reti neurali convoluzionali} (CNN).


\bibliographystyle{IEEEtran}
\bibliography{main}

\end{document}
