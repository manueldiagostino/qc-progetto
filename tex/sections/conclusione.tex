\section{Conclusione}\label{sec:conclusione}

Il progetto ha esplorato l'applicazione del \emph{Quantum Hadamard Edge
Detection}, tecnica di rilevamento dei bordi che utilizza i principi della
computazione quantistica e un innovativo approccio alla rappresentazione delle
immagini, la \emph{Quantum Probability Image Encoding}.

I risultati ottenuti mostrano il successo dell'esperimento condotto. In
particolare ne è stata evidenziata l'efficienza, sia in termini di spazio
necessario alla rappresentazione, sia per quanto riguarda il costo
computazionale richiesto.

L’utilizzo della libreria Qiskit ha permesso di implementare e testare il
modello su simulatori quantistici, fornendo un riscontro pratico sulla
fattibilità della tecnica. Tuttavia, a causa della complessità dei circuiti
transpilati, anche per immagini di piccole dimensioni, non è stato possibile
eseguire simulazioni con rumore per immagini più grandi di $2\times2$. 

Questo limite evidenzia la sfida rappresentata dall’elevato numero di porte
richieste per l’elaborazione dell’informazione quantistica e la necessità di
ottimizzare la profondità dei circuiti per migliorare la scalabilità
dell’algoritmo. Per superare le attuali barriere, futuri sviluppi potrebbero
concentrarsi sulla riduzione della complessità circuitale o esplorare strategie
ibride che combinino operazioni classiche e quantistiche.
