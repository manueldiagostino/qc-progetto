\section{Risultati}\label{sec:risultati}

\subsection{Esecuzione ideale}

Nel caso ideale, l'algoritmo è stato eseguito
utilizzando il \texttt{statevector\_simulator} di Qiskit, che consente di
simulare un sistema quantistico ideale senza la presenza di rumore.

L'immagine originale, rappresentata in Fig.~\ref{fig:16x16-original}, è stata
elaborata tramite il simulatore, producendo il risultato mostrato in
Fig.~\ref{fig:16x16-no-noise}. Questo risultato evidenzia chiaramente i bordi
dell'immagine con un'elevata precisione, grazie all'assenza di fattori
disturbanti come il rumore o le imperfezioni hardware. Questo scenario
rappresenta il limite teorico dell'algoritmo, fornendo una base di confronto per
valutare le prestazioni in condizioni più realistiche.

\begin{figure}
	\begin{center}
		\includegraphics[width=0.95\columnwidth]{16x16-original.png}
	\end{center}
	\caption{Immagine originale.}\label{fig:16x16-original}
\end{figure}

\begin{figure}
	\begin{center}
		\includegraphics[width=0.95\columnwidth]{16x16-no-noise.png}
	\end{center}
	\caption{Risultato dell'elaborazione tramite \texttt{statevector\_simulator}.}\label{fig:16x16-no-noise}
\end{figure}

\subsection{Esecuzione con rumore simulato}

Per valutare il comportamento dell'algoritmo in condizioni più vicine alla
realtà, è stato introdotto un modello di rumore (\texttt{NoiseModel}) basato
sull'hardware reale \texttt{ibm\_kyiv}. In questo caso, l'esecuzione quantistica
tiene conto di fattori come decoerenza e errori di lettura, che sono comuni nei
dispositivi quantistici attuali.

La Fig.~\ref{fig:16x16-shots} mostra il risultato dell'elaborazione con
l'inclusione del rumore, variando il numero di \textit{shots} (ossia il numero
di ripetizioni dell'esecuzione del circuito quantistico). Si osserva che un
numero maggiore di shots migliora il rilevamento dei bordi; inoltre, il
risultato finale non mostra particolari differenze rispetto al caso ideale. È
interessante notare che è possibile ottenere una buona approssimazione dei
contorni dell'immagine già a partire da 4096 shots.

\subsection{Esecuzione del circuito transpilato}

\begin{figure}
	\begin{center}
		\includegraphics[width=0.95\columnwidth]{16x16-shots.png}
	\end{center}
	\caption{Risultato dell'elaborazione utilizzando \texttt{NoiseModel} e numero di shots differenti.}\label{fig:16x16-shots}
\end{figure}
