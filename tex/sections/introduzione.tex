\section{Introduzione}

L'identificazione dei bordi è una tecnica fondamentale nell'elaborazione delle
immagini, utilizzata per individuare i contorni degli oggetti e le variazioni di
intensità in una scena. Questa metodologia rappresenta una componente cruciale
in numerosi ambiti, dalla computer vision alla robotica, fino all'analisi medica
delle immagini. Nonostante i progressi significativi nell'elaborazione classica
delle immagini, l'aumento della risoluzione e della complessità dei dati visivi
ha portato a sfide computazionali sempre maggiori, rendendo spesso i metodi
tradizionali onerosi in termini di tempo e risorse.

Nei primi anni '60, i filtri di Sobel \cite{SobelFeldman1968IsotropicGradient} e
Prewitt furono introdotti come i primi metodi strutturati per il rilevamento dei
bordi. Entrambi basati su operatori convolutivi, questi algoritmi utilizzano
maschere\footnote{Con il termine \textit{maschera} o \textit{kernel} di convoluzione 
si fa riferimento ad una piccola griglia sovrapposta in maniera iterativa a tutti i pixel
dell'immagine, aggiornando i valori in base ai primi vicini.} discrete per approssimare 
il gradiente di intensità in un'immagine,
rilevando così variazioni significative nei livelli di grigio. Sebbene semplici
ed efficienti, essi risultano sensibili al rumore e con conseguente difficoltà
nel gestire bordi sfumati. Negli anni '80, l'algoritmo di
Canny~\cite{CannyPaper} rappresentò una svolta significativa grazie
all'introduzione di un approccio più sofisticato al rilevamento dei bordi;
ancora oggi, rimane uno tra i metodi più utilizzati. Con l'avanzare della
tecnologia e l'aumento della potenza computazionale, il rilevamento dei bordi ha
beneficiato dell'utilizzo di tecniche basate sull'intelligenza artificiale, come
le \emph{reti neurali convoluzionali} (CNN). Soltanto recentemente
l'elaborazione quantistica delle immagini ha iniziato a emergere come un campo
innovativo e promettente, aprendo la strada a potenziali accelerazioni
esponenziali.

In questo progetto sarà presentata un'applicazione del \emph{Quantum Hadamard
Edge Detection} (QHED) \cite{Yao_2017}. La Sez. \ref{sec:background}
offre una panoramica sulle attuali tecniche di
rappresentazione quantistica delle immagini e una disamina delle tecniche
utilizzate nell'esperimento. La Sez. \ref{sec:implementazione}
è invece dedicata all'implementazione della soluzione proposta,
utilizzando la libreria \texttt{Qiskit} \cite{Qiskit}. In ultimo, sono
analizzati i risultati (Sez. \ref{sec:risultati}).
