\documentclass{beamer}
\usepackage{lmodern}
\usepackage{anyfontsize}
\usepackage[T1]{fontenc}
\usepackage[italian]{babel}

\usetheme[progressbar=head]{moloch}

\setbeamertemplate{title page}{
\begin{minipage}[c][0.9\paperheight]{\textwidth} % Altezza ridotta al 90%
	\centering

	\ifx\inserttitlegraphic\@empty\else\usebeamertemplate*{title graphic}\fi
	\vfill%

	\ifx\inserttitle\@empty\else {\usebeamertemplate*{title}}\fi
	\ifx\insertsubtitle\@empty\else {\usebeamertemplate*{subtitle}}\fi

	\usebeamertemplate*{title separator}

	{
	\centering
	\ifx\beamer@shortauthor\@empty\else\usebeamertemplate*{author}\fi
	\ifx\insertdate\@empty\else\usebeamertemplate*{date}\fi
	\ifx\insertinstitute\@empty\else\usebeamertemplate*{institute}\fi
	}

	\vfill
	\vspace*{1mm}
\end{minipage}
}

\setbeamertemplate{title}{
%  \raggedright%  % <-- Comment here
  \linespread{1.0}%
  \inserttitle%
  \par%
  \vspace*{0.5em}
}

\setbeamertemplate{author}{
  \insertauthor%
  \par%
	\vspace*{0.5em}
}
%
% \setbeamerfont{footnote}{size=\small}
% \setbeamercolor{footnote}{fg=blue, bg=lightgray}
% \setbeamertemplate{footnote page}{%
%     \begin{centering}
%         \vspace{1em} \footnotesize % Puoi cambiare la dimensione del font
%         \insertfootnotetext
%     \end{centering}
% }
%
% \setbeamerfont{footnote}{size=\small}


\graphicspath{{./images}}

\usepackage[style=numeric, sorting=none]{biblatex} % Per la gestione di BibTeX
\addbibresource{slides.bib}
\usepackage{csquotes}

\AtBeginSection[]
{
  \begin{frame}
    \frametitle{Sommario}
    \tableofcontents[currentsection]
  \end{frame}
}

\AtBeginSubsection[]{
  \begin{frame}
	\frametitle{Sommario}
	\tableofcontents[currentsection,currentsubsection]
	\end{frame}
}

\usepackage{mathtools}
\usepackage{amsmath}
\newcommand{\bigo}[1]{\ensuremath{O\left({#1}\right)}}
\usepackage{booktabs}
\DeclarePairedDelimiter\bra{\langle}{\rvert}
\DeclarePairedDelimiter\ket{\lvert}{\rangle}
\DeclarePairedDelimiterX\braket[2]{\langle}{\rangle}{#1\,\delimsize\vert\,\mathopen{}#2}
\usepackage{amsmath, amssymb}

\usepackage{doi}
\usepackage{subfig}

\usepackage{listings}
\usepackage{xcolor}
\definecolor{keywordcolor}{rgb}{0.1, 0.1, 0.75}  % Blue for keywords
\definecolor{stringcolor}{rgb}{0.1, 0.5, 0.1}    % Green for strings
\definecolor{commentcolor}{rgb}{0.5, 0.5, 0.5}   % Gray for comments
\definecolor{bgcolor}{rgb}{0.95, 0.95, 0.95}

\lstset{
    language=Python,                    % Set the language to Python
    backgroundcolor=\color{bgcolor},   % Background color for code
    basicstyle=\ttfamily\tiny, 
	keywordstyle=\color{keywordcolor}\bfseries, % Keywords in bold blue
    stringstyle=\color{stringcolor},   % Strings in green
    commentstyle=\color{commentcolor}\itshape, % Comments in italic gray
    identifierstyle=,                  % Default identifier style
    showstringspaces=false,            % Don't show spaces in strings
    numbers=left,                      % Line numbers on the left
    numberstyle=\tiny\color{commentcolor}, % Line number style
    numbersep=5pt, % Spaziatura dei numeri di riga
    %  frame=single, % Cornice intorno al blocco di codice
    % framerule=0.1pt,
    linewidth=0.98\columnwidth,
    stepnumber=1,                      % Line numbers increment by 1
    tabsize=4,                         % Tab size
    breaklines=true,                   % Automatically break long lines
    breakatwhitespace=true,            % Break lines only at whitespace
    frame=none,                      % Draw a box around the code
    captionpos=b,                      % Caption position: b for bottom
    escapeinside={(*@}{@*)},           % Escape to LaTeX between (*@ and @*)
}

\usepackage{caption} % Pacchetto per personalizzare le caption

% Imposta lo stile delle caption per le figure
\captionsetup[figure]{
    justification=centering, % Centra la caption
    font=small, % Dimensione del font (opzionale)
}

\usepackage[table,xcdraw]{xcolor}

%%%%%%%%%%%%%%%%%%%%%%%%%%%%%%%%%%%%%%%%%%%%%%%%%%%%%%%%%%%%%5


\title{Quantum Hadamard\\Edge Detection}
\author{Manuel Di Agostino}
\institute{Università di Parma}
\date{12 febbraio 2025}


\begin{document}
	\begin{frame}
		\titlepage
	\end{frame}
	\begin{frame}{Sommario}
		\tableofcontents
	\end{frame}
	
	\section{Background}\label{sec:background}

\subsection{Soluzioni classiche}
Le tecniche classiche per la rilevazione dei contorni prevedono l'utilizzo di 
kernel specifici, che permettono di calcolare nuovi valori di intensità per i pixel 
dell'immagine. Tra i metodi più famosi vi è sicuramente l'operatore di Sobel, 
che si può descrivere tramite l'applicazione di 2 kernel all'immagine originale:
\[
\mathbf{G_x} =
\begin{bmatrix}
+1 & 0 & -1 \\
+2 & 0 & -2 \\
+1 & 0 & -1
\end{bmatrix}
, \, 
\mathbf{G_y} =
\begin{bmatrix}
+1 & +2 & +1 \\
0 & 0 & 0 \\
-1 & -2 & -1
\end{bmatrix}
\]

Un'altra opzione, forse tra le più utilizzate al giorno d'oggi, è l'operatore di Canny.
Questo metodo ha un funzionamento del tutto analogo al precedente, ma aggiunge 
meccanismi per la riduzione del rumore nell'immagine \cite{digital_image_processing}.
La complessità di queste tecniche è lineare rispetto al numero di pixel totali 
dell'immagine, dato che è necessaria una visita completa. 

Per un'immagine $M \times L = N$, si utilizzano $n$ bit per enumerare i pixel dell'immagine
in formato binario, dove $N = 2^n$, ottenendo una complessità rispetto ai bit esponenziale $\bigo{2^n}$.
In questo progetto verrà mostrato come, dopo una prima fase di preparazione, è possibile risolvere
il problema in tempo costante $\bigo{1}$.

\subsection{Sistemi quantistici}

Analogamente a quanto accade nei computer classici, i computer quantistici utilizzano i \textbf{quantum bit}, chiamati \emph{qubit}. I qubit rappresentano la più piccola unità di informazione e sono implementati attraverso sistemi quantistici bidimensionali. Le quantità fisiche comunemente usate per questo scopo includono lo spin di una particella o gli stati eccitati degli atomi.

Assemblando più qubit, è possibile costruire sistemi quantistici la cui dinamica è descritta da spazi vettoriali complessi. Un sistema composto da un singolo qubit è completamente descritto da
\begin{equation}
	\ket{\psi} = \begin{bmatrix}
		\alpha\\ \beta
	\end{bmatrix} = \alpha \ket{0} + \beta \ket{1},\quad \alpha,\beta \in
	\mathbb{C}
	\label{eq:1-qubit-state}
\end{equation}

Mentre un bit classico può assumere soltanto uno tra due possibili valori
(generalmente 0 e 1), un bit quantistico è denotato da una combinazione lineare
dei suoi stati base, pesata dai coefficienti complessi $\alpha$ e $\beta$. Tali
coefficienti sono detti \emph{ampiezze di probabilità} e rispettano la seguente:
\begin{equation}
	|\alpha|^2 + |\beta|^2 = 1
	\label{eq:qubit-prob}
\end{equation}

Per descrivere lo stato di un sistema quantistico composto da più qubit, è
necessario effettuare un'operazione chiamata \emph{prodotto tensoriale} tra i
singoli stati coinvolti. Ad esempio, considerati i vettori di stati
\[
\ket{\psi_1} = \begin{bmatrix} a_1 \\ a_2 \end{bmatrix}, \quad \ket{\psi_2} = \begin{bmatrix} b_1 \\ b_2 \end{bmatrix},
\]
il loro prodotto tensore è definito come:
\begin{equation}
\ket{\psi_1} \otimes \ket{\psi_2}
= \begin{bmatrix}
a_1 \begin{bmatrix} b_1 \\ b_2 \end{bmatrix} \\
a_2 \begin{bmatrix} b_1 \\ b_2 \end{bmatrix}
\end{bmatrix} =
\begin{bmatrix}
a_1 b_1 \\ a_1 b_2 \\ a_2 b_1 \\ a_2 b_2
\end{bmatrix}.
\label{eq:tensor-prod}
\end{equation}
Equivalentemente, può essere scritto come $\ket{\psi_1}\ket{\psi_2} \text{ o } \ket{\psi_1
\psi_2}$.

\todo{Se serve, parlare dell'entanglement}

\subsection{Quantum image processing}

\subsubsection*{Rappresentazione dei pixel}
\subsubsection*{QHED}

\subsection{Modellazione del rumore}

	\section{Implementazione}

\subsection{Creazione del circuito}

\begin{frame}[fragile]{Normalizzazione dell'immagine}
	\begin{itemize}
		\item Normalizazione dell'immagine
		\begin{lstlisting}
def amplitude_encode(img_data):
	# Calculate the RMS value
	rms = np.sqrt(np.sum(np.sum(img_data**2, axis=1)))
	# Create normalized image
	image_norm = []
	for arr in img_data:
	for ele in arr:
	image_norm.append(ele / rms)
	# Return the normalized image as a numpy array
	return np.array(image_norm)

# Get the amplitude ancoded pixel values
image_norm_h = amplitude_encode(image)
		\end{lstlisting}
	\end{itemize}
\end{frame}

\begin{frame}[fragile]{Il circuito}
	\begin{itemize}
		\item Strutturazione
		\begin{lstlisting}[language=Python]
from qiskit import *
from qiskit import transpile
from qiskit_aer import Aer

# Create the circuit for horizontal scan
qc_h = QuantumCircuit(total_qb)
qc_h.initialize(image_norm_h, range(1, total_qb))
qc_h.h(0)
qc_h.unitary(D2n_1, range(total_qb))
qc_h.h(0)
		\end{lstlisting}
	\end{itemize}
\end{frame}

\begin{frame}[fragile]
	\begin{itemize}
		\item Immagine di esempio 16x16 in input
		\begin{figure}
			\centering
			\subfloat[Immagine originale]{%
			\includegraphics[width=0.4\textwidth]{16x16-original.png}%
			\label{fig:16x16-original}
			}
			\hfill
			\subfloat[Immagine normalizzata]{%
			\includegraphics[width=0.5\textwidth]{16x16.png}%
			\label{fig:16x16-norm}
			}
			\caption{Confronto tra l'immagine originale (\ref{fig:16x16-original}) e l'immagine normalizzata (\ref{fig:16x16-norm}).}
		\end{figure}
	\end{itemize}
\end{frame}

\begin{frame}
	\begin{itemize}
		\item Circuito risultante
			\begin{figure}
				\centering
				\includegraphics[width=0.65\textwidth]{16x16-circuit.png}
			\end{figure}
	\end{itemize}
\end{frame}

\begin{frame}{Codifica tramite gate}
	\begin{itemize}
		\item Per un'immagine 2x2
			\begin{figure}
				\centering
				\includegraphics[width=0.8\textwidth]{2x2-transpiled.png}
			\end{figure}
	\end{itemize}
\end{frame}


\subsection{Error handling}

\begin{frame}[fragile]{La gestione degli errori}
	\begin{itemize}
		\item È stata utilizzata la classe \texttt{NoiseModel} di Qiskit
		\item Dati estrapolati dal backend reale \texttt{ibm\_kyiv}
		\begin{lstlisting}
from qiskit_aer import AerSimulator
from qiskit_aer.noise import NoiseModel

from qiskit_ibm_runtime import QiskitRuntimeService
service = QiskitRuntimeService(channel="ibm_quantum", token="XXX")

backend = service.backend("ibm_kyiv")
from qiskit_aer.noise import (NoiseModel, QuantumError, ReadoutError,
    pauli_error, depolarizing_error, thermal_relaxation_error)

noise_model = NoiseModel.from_backend(backend)
# Get coupling map from backend
coupling_map = backend.configuration().coupling_map
# Get basis gates from noise model
basis_gates = noise_model.basis_gates
		\end{lstlisting}
	\end{itemize}
\end{frame}


	\section{Risultati}\label{sec:risultati}

\subsection{Esecuzione ideale}

Nel caso ideale, l'algoritmo è stato eseguito
simulando il sistema quantistico tramite \texttt{statevector\_simulator}, 
senza la presenza di rumore. L'immagine originale,
elaborata tramite il simulatore, produce il risultato mostrato in
Fig.~\ref{fig:statevector-detection}. Questo risultato evidenzia chiaramente i bordi
dell'immagine con un'elevata precisione, grazie all'assenza di fattori
disturbanti come il rumore o le imperfezioni hardware. Questo scenario
rappresenta il limite teorico dell'algoritmo, fornendo una base di confronto per
valutare le prestazioni in condizioni più realistiche.

\begin{figure}
	\centering
	\includegraphics[width=0.45\textwidth]{statevector-detection.png}
	\caption{Risultato dell'elaborazione tramite \texttt{state\_vector}.}
	\label{fig:statevector-detection}
\end{figure}

\subsection{Esecuzione con rumore simulato}

Per valutare il comportamento dell'algoritmo in condizioni più vicine alla
realtà, è stato introdotto un modello di rumore (\texttt{NoiseModel}) basato
sull'hardware reale \texttt{ibm\_kyiv}. In questo caso, l'esecuzione quantistica
tiene conto di fattori come decoerenza e errori di lettura, che sono comuni nei
dispositivi quantistici attuali.

La Fig.~\ref{fig:16x16-shots} mostra il risultato dell'elaborazione con
l'inclusione del rumore, variando il numero di \textit{shots} (ossia il numero
di ripetizioni dell'esecuzione del circuito quantistico). Si osserva che un
numero maggiore di shots migliora il rilevamento dei bordi; inoltre, il
risultato finale non mostra particolari differenze rispetto al caso ideale. È
interessante notare che è possibile ottenere una buona approssimazione dei
contorni dell'immagine già a partire da 4096 shots.

\begin{figure}[ht]
	\centering
	\includegraphics[width=0.9\columnwidth]{16x16-shots.png}
	\caption{Risultato dell'elaborazione utilizzando \texttt{NoiseModel} con numero di shots differenti.}\label{fig:16x16-shots}
\end{figure}

\subsection{Esecuzione del circuito ``transpiled''}
È stato possibile eseguire il transpiling soltanto di circuiti relativi a
immagini $2 \times 2$. Utilizzando il
parametro \texttt{optimization\_level=3}, la versione ``transpiled'' del circuito
rappresentato in Fig.~\ref{fig:circuito-2x2} risulta avere profondità 64. 
In Fig. \ref{fig:2x2-simulation} e \ref{fig:2x2-shots} viene mostrata 
l'immagine originale, seguita dalle rispettive simulazioni, con e senza rumore.
Viste le dimensioni ridotte dell'immagine, l'esecuzione
con diverse soglie di shots non sembra essere particolarmente significativa.
Nonostante questo, si può dire che il comportamento del circuito è corretto, 
poiché i risultati coincidono con le misurazioni ottenute tramite \texttt{state\_vector}.

\begin{figure}[ht]
	\centering
	\includegraphics[width=0.9\columnwidth]{2x2-simulation.png}
	\caption{Esempio di immagine $2\times2$.}\label{fig:2x2-simulation}
\end{figure}

\begin{figure}[ht]
	\centering
	\includegraphics[width=0.9\columnwidth]{2x2-shots.png}
	\caption{Risultato dell'elaborazione sul circuito transpiled, 
	utilizzando \texttt{NoiseModel} con numero di shots differenti 
	su immagine $2\times2$.}\label{fig:2x2-shots}
\end{figure}

\subsection{Complessità spaziale e temporale}

In generale, gli algoritmi classici di rilevamento dei bordi si basano sulla
computazione dei gradienti dell'immagine; questo significa dover identificare
delle transizioni di intensità per ogni pixel. Data $N=2^n, n \in
\mathbb{N}$ la dimensione dell'immagine in pixel, metodi analoghi a quello di Sobel
hanno una complessità temporale nell'ordine di $\bigo{2^n}$, con aumento
lineare rispetto al numero di pixel presenti. L'algoritmo QSobel risulta
migliore ($\bigo{n^2}$), utilizzando la rappresentazione FRQI e
richiedendo un totale di $2n+1$ qubits \cite{qhed_frqi}.

L'algoritmo utilizzato in questo progetto utilizza invece soltanto $n=\log_2{N}$
qubits, permettendo un miglioramento considerevole in termini del costo spaziale
richiesto. Inoltre, escludendo la fase di preparazione del circuito, la
procedura di calcolo dei gradienti tramite \emph{decrement gate} è eseguita in
$\bigo{1}$, soglia nettamente inferiore rispetto al QSobel. Per quanto riguarda
la fase di preparazione, lo schema QPIE richiede $\bigo{n^2}$ passi nel caso
peggiore, leggermente superiori a quelli della FRQI ($\bigo{n}+\bigo{\log^2{n}}$).



	\begin{frame}[allowframebreaks]{Bibliografia}
		\printbibliography
	\end{frame}

\end{document}
